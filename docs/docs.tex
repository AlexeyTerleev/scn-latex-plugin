\documentclass{scndocument}

\usepackage{scn}

\begin{document}
\begin{SCn}
  \ActivateBG

  \scnheader{\LaTeX plugin for SCn-code representation}
  \scnidtf{scn-latex-plugin}

  \scntext{note}{Current documentation is an example of use of described macros}

  \scnsuperset{SCn macro}

  \scnheader{SCn macro}
  \begin{scnrelfromset}{subdividing}
    \scnitem{SCn function}
    \begin{scnindent}
      \begin{scnrelfromset}{subdividing by puprose}
        \scnitem{structure and table of contents}
        \scnitem{headers and style}
        \scnitem{basic relation}
      \end{scnrelfromset}
    \end{scnindent}
    \scnitem{SCn environment}
    \begin{scnindent}
      \begin{scnrelfromset}{subdividing by puprose}
        \scnitem{advanced structure}
        \scnitem{advanced relation}
      \end{scnrelfromset}
    \end{scnindent}
    \scnitem{scn-latex-plugin variable}
  \end{scnrelfromset}

  \scnheader{structure and table of contents}
  \scntext{note}{names of sections differ in table of contents depth, but in the text of \textit{PDF} they all are called \scnqq{Section}}
  \begin{scnhaselementset}
    \scnitem{scseparatedfragment}
    \scnitem{scsectionfamily}
    \scnitem{scchapter}
    \scnitem{scsection}
    \scnitem{scsubsection}
    \scnitem{scsubsubsection}
    \scnitem{scparagraph}
    \scnitem{currentname}
    \begin{scnindent}
      \scntext{puprose}{print name of the current section}
    \end{scnindent}
  \end{scnhaselementset}

  \scnheader{headers and style}
  \begin{scnhaselementset}
    \scnitem{scnheader}
    \begin{scnindent}
      \scntext{purpose}{create header that represents sc-element}
    \end{scnindent}
    \scnitem{scheaderlocal}
    \begin{scnindent}
      \scntext{purpose}{works same as \scnkeyword{scnheader}, but without resetting indents}
    \end{scnindent}
    \scnitem{scnstructheader}
    \begin{scnindent}
      \scntext{purpose}{create header that represents sc-structure}
    \end{scnindent}
    \scnitem{scnstructheaderlocal}
    \begin{scnindent}
      \scntext{purpose}{works same as \scnkeyword{scnstructheader}, but without resetting indents}
    \end{scnindent}
    \scnitem{scnstructidtf}
    \begin{scnindent}
      \scntext{purpose}{style for struct identifiers inside files of ostis-systems}
    \end{scnindent}
    \scnitem{scnkeyword}
    \begin{scnindent}
      \scntext{purpose}{style for keywords inside files of ostis-systems}
    \end{scnindent}
    \scnitem{scnsectionheader}
    \begin{scnindent}
      \scntext{purpose}{create header that represents section}
    \end{scnindent}
    \scnitem{scnsegmentheader}
    \begin{scnindent}
      \scntext{purpose}{create header that represents segment}
    \end{scnindent}
    \scnitem{scnfilelong}
    \begin{scnindent}
      \scntext{purpose}{create sc-element that represents \scnkeyword{file of the ostis-system}}
    \end{scnindent}
    \scnitem{scnfileimage}
    \begin{scnindent}
      \scntext{purpose}{create sc-element that represents \scnkeyword{file of the ostis-system} as image}
    \end{scnindent}
    \scnitem{scnfiletable}
    \begin{scnindent}
      \scntext{purpose}{create sc-element that represents \scnkeyword{file of the ostis-system} as table}
    \end{scnindent}
    \scnitem{scnfileclass}
    \scnitem{scnnonamednode}
    \begin{scnindent}
      \scntext{purpose}{create sc-element without sc-identifier}
    \end{scnindent}
    \scnitem{scnsourcecomment}
    \begin{scnindent}
      \scntext{purpose}{create comment in knowledge base source code}
    \end{scnindent}
    \scnitem{scnendfragmentcomment}
    \begin{scnindent}
      \scntext{purpose}{create comment in knowledge base source code}
      \scntext{note}{used at the end of any fragment (e.g. segment, section and so on)}
    \end{scnindent}
    \scnitem{scnendcurrentsectioncomment}
    \begin{scnindent}
      \scntext{purpose}{create comment in knowledge base source code}
      \scntext{note}{used at the end of sections}
    \end{scnindent}
    \scnitem{scnendsegmentcomment}
    \begin{scnindent}
      \scntext{purpose}{create comment in knowledge base source code}
      \scntext{note}{used at the end of segments}
    \end{scnindent}
    \scnitem{scnqq}
    \begin{scnindent}
      \scntext{purpose}{alias for creating quotes of the first level}
    \end{scnindent}
    \scnitem{scnqqi}
    \begin{scnindent}
      \scntext{purpose}{alias for creating quotes of the second level}
    \end{scnindent}
  \end{scnhaselementset}

  \scnheader{basic relation}
  \begin{scnhaselementset}
    \scnitem{scnrelfrom}
    \begin{scnindent}
      \scntext{purpose}{create constant positive relation from previous header}
    \end{scnindent}
    \scnitem{scnrelsuperset}
    \begin{scnindent}
      \scntext{purpose}{create subset of previous header with some relation}
    \end{scnindent}
    \scnitem{scnvarrelfrom}
    \begin{scnindent}
      \scntext{purpose}{create variable positive relation from previous header}
    \end{scnindent}
    \scnitem{scnrelto}
    \begin{scnindent}
      \scntext{purpose}{create constant positive relation to previous header}
    \end{scnindent}
    \scnitem{scnvarrelto}
    \begin{scnindent}
      \scntext{purpose}{create variable positive relation to previous header}
    \end{scnindent}
    \scnitem{scnrelboth}
    \begin{scnindent}
      \scntext{purpose}{create constant positive relation from and to previous header}
    \end{scnindent}
    \scnitem{scneq}
    \begin{scnindent}
      \scntext{purpose}{used for semantic equivalence relation for previous header}
    \end{scnindent}
    \scnitem{scneqfile}
    \begin{scnindent}
      \scntext{purpose}{used for semantic equivalence relation for previous header}
      \scntext{note}{for use with files as second domain}
    \end{scnindent}
    \scnitem{scneqimage}
    \begin{scnindent}
      \scntext{purpose}{used for semantic equivalence relation for previous header}
      \scntext{note}{for use with images as second domain}
    \end{scnindent}
    \scnitem{scneqtable}
    \begin{scnindent}
      \scntext{purpose}{used for semantic equivalence relation for previous header}
      \scntext{note}{for use with tables as second domain}
    \end{scnindent}
    \scnitem{scneqfileclass}
    \scnitem{scnsubset}
    \begin{scnindent}
      \scntext{purpose}{create superset for previous header}
    \end{scnindent}
    \scnitem{scnnotsubset}
    \begin{scnindent}
      \scntext{purpose}{mark that previous header is not a subset of some set}
    \end{scnindent}
    \scnitem{scnsuperset}
    \begin{scnindent}
      \scntext{purpose}{create a subset for previous header}
    \end{scnindent}
    \scnitem{scniselement}
    \begin{scnindent}
      \scntext{purpose}{mark previous header as a constant element of some set}
    \end{scnindent}
    \scnitem{scnisvarelement}
    \begin{scnindent}
      \scntext{purpose}{mark previous header as a variable element of some set}
    \end{scnindent}
    \scnitem{scnhaselement}
    \begin{scnindent}
      \scntext{purpose}{mark that previous header has some constant element}
    \end{scnindent}
    \scnitem{scnhasvarelement}
    \begin{scnindent}
      \scntext{purpose}{mark that previous header has some variable element}
    \end{scnindent}
    \scnitem{scnhaselementrole}
    \begin{scnindent}
      \scntext{purpose}{mark that previous header has some constant element with role in it}
    \end{scnindent}
    \scnitem{scnhasvarelementrole}
    \begin{scnindent}
      \scntext{purpose}{mark that previous header has some variable element with role in it}
    \end{scnindent}
    \scnitem{scniselementrole}
    \begin{scnindent}
      \scntext{purpose}{mark previous header as a constant element of some set with role in it}
    \end{scnindent}
    \scnitem{scnidtf}
    \begin{scnindent}
      \scntext{purpose}{create identifier of previous header}
    \end{scnindent}
    \scnitem{scnidtftext}
    \begin{scnindent}
      \scntext{purpose}{create identifier of previous header with dome relation}
    \end{scnindent}
    \scnitem{scntext}
    \begin{scnindent}
      \scntext{purpose}{create relation from previous header to some \scnkeyword{file of the ostis-system}}
    \end{scnindent}
  \end{scnhaselementset}

  \scnheader{advanced structure}
  \begin{scnhaselementset}
    \scnitem{SCn}
    \begin{scnindent}
      \scntext{purpose}{wraps all \scnkeyword{sc.n-text}}
    \end{scnindent}
    \scnitem{scnindent}
    \begin{scnindent}
      \scntext{purpose}{create new indent to make all following connectors until the end of environment incident with the previous header}
    \end{scnindent}
    \scnitem{scnitemize}
    \begin{scnindent}
      \scntext{purpose}{create itemized list inside \scnkeyword{file of the ostis-system} of level 1}
    \end{scnindent}
    \scnitem{scnitemizeii}
    \begin{scnindent}
      \scntext{purpose}{create itemized list inside \scnkeyword{file of the ostis-system} of level 2}
    \end{scnindent}
    \scnitem{scnitemizeiii}
    \begin{scnindent}
      \scntext{purpose}{create itemized list inside \scnkeyword{file of the ostis-system} of level 3}
    \end{scnindent}
    \scnitem{scnenumerate}
    \begin{scnindent}
      \scntext{purpose}{create enumerated list inside \scnkeyword{file of the ostis-system}}
    \end{scnindent}
    \scnitem{scnstruct}
    \begin{scnindent}
      \scntext{purpose}{create \scnkeyword{sctructure} to specify its content inside environment}
    \end{scnindent}
    \scnitem{scnsubstruct}
    \begin{scnindent}
      \scntext{purpose}{this macros is similar to \scnkeyword{scnstruct} but represents immersion operation semantics}
    \end{scnindent}
    \scnitem{scnset}
    \begin{scnindent}
      \scntext{purpose}{create \scnkeyword{sc-set} to specify its content inside environment}
    \end{scnindent}
    \scnitem{scnvector}
    \begin{scnindent}
      \scntext{purpose}{create oriented \scnkeyword{sc-set} to specify its content inside environment}
    \end{scnindent}
    \scnitem{scnhierstruct}
  \end{scnhaselementset}
  
  \scnheader{advanced relation}
  \begin{scnhaselementset}
    \scnitem{scnitem}
    \begin{scnindent}
      \scntext{purpose}{create item in \scnkeyword{advanced relation} environments}
    \end{scnindent}
    \scnitem{scnfileitem}
    \begin{scnindent}
      \scntext{purpose}{create item in \scnkeyword{advanced relation} environments represented with \scnkeyword{file of the ostis-system}}
    \end{scnindent}
    \scnitem{scnrelfromlist}
    \begin{scnindent}
      \scntext{purpose}{create constant positive relation from previous header to list of \scnkeyword{sc-elements}}
    \end{scnindent}
    \scnitem{scnreltolist}
    \begin{scnindent}
      \scntext{purpose}{create constant positive relation to previous header from list of \scnkeyword{sc-elements}}
    \end{scnindent}
    \scnitem{scnrelbothlist}
    \begin{scnindent}
      \scntext{purpose}{create constant positive relation from previous header to list of \scnkeyword{sc-elements} and vice versa}
    \end{scnindent}
    \scnitem{scnrelfromset}
    \begin{scnindent}
      \scntext{purpose}{create constant positive relation from previous header to unoriented \scnkeyword{sc-set} containing list of \scnkeyword{sc-elements}}
    \end{scnindent}
    \scnitem{scnrelfromvector}
    \begin{scnindent}
      \scntext{purpose}{create constant positive relation from previous header to oriented \scnkeyword{sc-set} containing list of \scnkeyword{sc-elements}}
    \end{scnindent}
    \scnitem{scnhaselementrolelist}
    \begin{scnindent}
      \scntext{purpose}{create list of elements included in previous header on specified role in it}
    \end{scnindent}
    \scnitem{scnreltoset}
    \begin{scnindent}
      \scntext{purpose}{create constant positive relation to previous header from unoriented \scnkeyword{sc-set} containing list of \scnkeyword{sc-elements}}
    \end{scnindent}
    \scnitem{scnreltovector}
    \begin{scnindent}
      \scntext{purpose}{create constant positive relation to previous header from oriented \scnkeyword{sc-set} containing list of \scnkeyword{sc-elements}}
    \end{scnindent}
    \scnitem{scnhaselementset}
    \begin{scnindent}
      \scntext{puprose}{mark that previous header has unoriented \scnkeyword{sc-set} containing list of \scnkeyword{sc-elements} as constant element}
    \end{scnindent}
    \scnitem{scnhaselementvector}
    \begin{scnindent}
      \scntext{puprose}{mark that previous header has oriented \scnkeyword{sc-set} containing list of \scnkeyword{sc-elements} as constant element}
    \end{scnindent}
    \scnitem{scneqtoset}
    \begin{scnindent}
      \scntext{purpose}{mark that previous header is semantically equivalent to unoriented \scnkeyword{sc-set} containing list of \scnkeyword{sc-elements}}
    \end{scnindent}
    \scnitem{scneqtovector}
    \begin{scnindent}
      \scntext{purpose}{mark that previous header is semantically equivalent to oriented \scnkeyword{sc-set} containing list of \scnkeyword{sc-elements}}
    \end{scnindent}
    \scnitem{scnhassubset}
    \begin{scnindent}
      \scntext{purpose}{mark that previous header includes unoriented \scnkeyword{sc-set} containing list of \scnkeyword{sc-elements}}
    \end{scnindent}
    \scnitem{scnhassubstruct}
    \begin{scnindent}
      \scntext{purpose}{this macros is similar to \scnkeyword{scnhassubset} but represents immersion operation semantics}
    \end{scnindent}
  \end{scnhaselementset}

  \scnheader{scn-latex-plugin variable}
  \begin{scnhaselementset}
    \scnitem{tabsize}
    \begin{scnindent}
      \scntext{purpose}{set length of indents}
    \end{scnindent}
    \scnitem{scnsupergroupsign}
    \begin{scnindent}
      \scntext{purpose}{sign to represent classes of classes}
    \end{scnindent}
    \scnitem{scnrolesign}
    \begin{scnindent}
      \scntext{purpose}{sign to represent role relations}
    \end{scnindent}
    \scnitem{language specific variables}
    \begin{scnindent}
        \scnhaselement{segment}
        \begin{scnindent}
          \scntext{purpose}{represent \scnqq{Segment} word in different languages}
        \end{scnindent}
        \scnhaselement{complfragm}
        \begin{scnindent}
          \scntext{purpose}{represent \scnqq{Completed} word in different languages}
        \end{scnindent}
        \scnhaselement{complsect}
        \begin{scnindent}
          \scntext{purpose}{represent \scnqq{Completed Section} in different languages}
        \end{scnindent}
        \scnhaselement{complsegm}
        \begin{scnindent}
          \scntext{purpose}{represent \scnqq{Completed Segment} in different languages}
        \end{scnindent}
        \scnhaselement{curpage}
        \begin{scnindent}
          \scntext{purpose}{represent \scnqq{page number} in different languages}
        \end{scnindent}
        \scnhaselement{sectfamily}
        \begin{scnindent}
          \scntext{purpose}{represent \scnqq{Section family} in different languages}
        \end{scnindent}
        \scnhaselement{sect}
        \begin{scnindent}
          \scntext{purpose}{represent \scnqq{Section} in different languages}
        \end{scnindent}
        \scnhaselement{bibref}
        \begin{scnindent}
          \scntext{purpose}{represent \scnqq{bibliographic reference} in different languages}
        \end{scnindent}
        \scnhaselement{annot}
        \begin{scnindent}
          \scntext{purpose}{represent \scnqq{annotation} in different languages}
        \end{scnindent}
    \end{scnindent}
  \end{scnhaselementset}

  \scnheader{\LaTeX plugin for SCn-code representation}
  \scnrelfrom{default representation language}{English language}
  \scnrelfrom{class of actions}{action. change plugin language}
  \begin{scnindent}
    \scnhaselement{action. change variable value}
    \begin{scnindent}
      \scnhaselementrole{1}{\scnnonamednode}
      \begin{scnindent}
        \begin{scneqtoset}
          \scnheader{language specific variables}
          \scnhasvarelement{varible}
        \end{scneqtoset}
      \end{scnindent}
      \scnhaselementrole{2}{new value}
      \scnrelfrom{result}{\scnnonamednode}
      \begin{scnindent}
        \begin{scneqtoset}
          \scnheader{language specific variables}
          \scnhasvarelement{varible}
          \begin{scnindent}
            \scnrelfrom{value}{new value}
          \end{scnindent}
        \end{scneqtoset}
      \end{scnindent}
      \scntext{explanation}{change values of all \scnkeyword{language specific variables} via renewcommand}
    \end{scnindent}
  \end{scnindent}

\newpage
\end{SCn}
\end{document}
